%!TEX TS-program = xelatex
%!TEX encoding = UTF-8 Unicode
% Awesome CV LaTeX Template for Cover Letter
%
% This template has been downloaded from:
% https://github.com/posquit0/Awesome-CV
%
% Authors:
% Claud D. Park <posquit0.bj@gmail.com>
% Lars Richter <mail@ayeks.de>
%
% Template license:
% CC BY-SA 4.0 (https://creativecommons.org/licenses/by-sa/4.0/)
%


%-------------------------------------------------------------------------------
% CONFIGURATIONS
%-------------------------------------------------------------------------------
% A4 paper size by default, use 'letterpaper' for US letter
\documentclass[11pt, a4paper]{awesome-cv}
\usepackage{xfrac}

% Configure page margins with geometry
\geometry{left=1.4cm, top=.8cm, right=1.4cm, bottom=1.8cm, footskip=.5cm}

% Specify the location of the included fonts
\fontdir[fonts/]

% Color for highlights
% Awesome Colors: awesome-emerald, awesome-skyblue, awesome-red, awesome-pink, awesome-orange
%                 awesome-nephritis, awesome-concrete, awesome-darknight
\colorlet{awesome}{awesome-red}
% Uncomment if you would like to specify your own color
% \definecolor{awesome}{HTML}{CA63A8}

% Colors for text
% Uncomment if you would like to specify your own color
% \definecolor{darktext}{HTML}{414141}
% \definecolor{text}{HTML}{333333}
% \definecolor{graytext}{HTML}{5D5D5D}
% \definecolor{lighttext}{HTML}{999999}

% Set false if you don't want to highlight section with awesome color
\setbool{acvSectionColorHighlight}{true}

% If you would like to change the social information separator from a pipe (|) to something else
\renewcommand{\acvHeaderSocialSep}{\quad\textbar\quad}


%-------------------------------------------------------------------------------
%	PERSONAL INFORMATION
%	Comment any of the lines below if they are not required
%-------------------------------------------------------------------------------
% Available options: circle|rectangle,edge/noedge,left/right
% \photo{./examples/profile.png}
\name{}{Ben Bubnick}
\position{Analytics Training Specialist{\enskip\cdotp\enskip}Adjunct Professor}
\address{3419 Superior Park Drive, Cleveland Heights, Ohio, 44118, U.S.}

\email{ben.bubnick@gmail.com}
%\homepage{www.benbubnick.com}
% \github{bubnicbf}
\linkedin{benbubnick}
% \gitlab{gitlab-id}
% \stackoverflow{SO-id}{SO-name}
% \twitter{@twit}
% \skype{skype-id}
% \reddit{reddit-id}
% \extrainfo{extra informations}

\quote{``10+ years experience physics/astronomy/analytics lecturer"}


%-------------------------------------------------------------------------------
%	LETTER INFORMATION
%	All of the below lines must be filled out
%-------------------------------------------------------------------------------
% The company being applied to
\recipient
  {Department of Physics}
  {John Carroll University\\1 John Carroll Boulevard\\University Heights, Ohio 44118}
% The date on the letter, default is the date of compilation
\letterdate{\today}
% The title of the letter
\lettertitle{Job Application for Adjunct Faculty, Physics}
% How the letter is opened
\letteropening{Dear Hiring Manager,}
% How the letter is closed
\letterclosing{Sincerely,}
% Any enclosures with the letter
\letterenclosure[Attached]{Resume}
% \letterenclosure[Attached]{Resume, Teaching Statement}


%-------------------------------------------------------------------------------
\begin{document}

% Print the header with above personal informations
% Give optional argument to change alignment(C: center, L: left, R: right)
\makecvheader[R]

% Print the footer with 3 arguments(<left>, <center>, <right>)
% Leave any of these blank if they are not needed
\makecvfooter
  {\today}
  {Ben Bubnick~~~·~~~Cover Letter}
  {}

% Print the title with above letter informations
\makelettertitle

%-------------------------------------------------------------------------------
%	LETTER CONTENT
%-------------------------------------------------------------------------------
\begin{cvletter}

\lettersection{Who I am}
I have been a data scientist with IBM for the last 4$\sfrac{1}{2}$ years and have worked as a scientist in some capacity over the last 13 years.  I have taught or tutored at the collegiate level over that same time period, most recently as a physics lecturer with Lorain County Community College (LCCC).  I volunteer with many scientific outreach programs, such as an after school computer science program for East side Cleveland City Schools, based on a program delivered by Hyland Software in the west side suburbs.

My scholastic training is broad and encompasses a number of the various areas: notably with physics, mathematics, analytics, software, and teaching.  My masters work focused on teaching along with research, something I have brought to many of my positions industry.  As noted in my resume, one of the roles I have filled at IBM has been in onboarding and cross-team training, utilizing software engineering in areas of data cleaning and data analytics in areas of machine learning algorithms.

\lettersection{Why JCU}
Though trained as a researcher, I define myself broadly as a teacher and am eager to return to the classroom. I am prepared to teach physics at many levels.  At Miami University, I taught a "Physics for Poets" style course, while at the same time teaching technical physics courses to both regular and remedial populations at Cincinnati State.  At LCCC I have taught both laboratory and lecture classes to a wide range of preparedness levels.  I am enthusiastic to bring that same experience to John Carroll University for that same reason--the diversity of the student population makes for a challenging environment that I thrive in.

\lettersection{Why Me?}
Not every student learns in the same way, especially with a subject as complex and daunting as physics.  These days the didactic teaching methods we all learned on have to share the stage with active engagement methods, which have been my primary teaching methods both in industry and academia.  I have long used Interactive Lecture Demonstrations teaching my lab course as you might expect, but have also noted significant improvement incorporating Interactive Computer Based Tutorials and Just in Time Teaching methods into my curriculum.  % I go into more depth on this in my Teaching Statement.

\end{cvletter}


%-------------------------------------------------------------------------------
% Print the signature and enclosures with above letter informations
\makeletterclosing

\end{document}
