%!TEX TS-program = xelatex
%!TEX encoding = UTF-8 Unicode
% Awesome CV LaTeX Template for Cover Letter
%
% This template has been downloaded from:
% https://github.com/posquit0/Awesome-CV
%
% Authors:
% Claud D. Park <posquit0.bj@gmail.com>
% Lars Richter <mail@ayeks.de>
%
% Template license:
% CC BY-SA 4.0 (https://creativecommons.org/licenses/by-sa/4.0/)
%


%-------------------------------------------------------------------------------
% CONFIGURATIONS
%-------------------------------------------------------------------------------
% A4 paper size by default, use 'letterpaper' for US letter
\documentclass[11pt, a4paper]{awesome-cv}
\usepackage{xfrac}

% Configure page margins with geometry
\geometry{left=1.4cm, top=.8cm, right=1.4cm, bottom=1.8cm, footskip=.5cm}

% Specify the location of the included fonts
\fontdir[fonts/]

% Color for highlights
% Awesome Colors: awesome-emerald, awesome-skyblue, awesome-red, awesome-pink, awesome-orange
%                 awesome-nephritis, awesome-concrete, awesome-darknight
\colorlet{awesome}{awesome-red}
% Uncomment if you would like to specify your own color
% \definecolor{awesome}{HTML}{CA63A8}

% Colors for text
% Uncomment if you would like to specify your own color
% \definecolor{darktext}{HTML}{414141}
% \definecolor{text}{HTML}{333333}
% \definecolor{graytext}{HTML}{5D5D5D}
% \definecolor{lighttext}{HTML}{999999}

% Set false if you don't want to highlight section with awesome color
\setbool{acvSectionColorHighlight}{true}

% If you would like to change the social information separator from a pipe (|) to something else
\renewcommand{\acvHeaderSocialSep}{\quad\textbar\quad}


%-------------------------------------------------------------------------------
%	PERSONAL INFORMATION
%	Comment any of the lines below if they are not required
%-------------------------------------------------------------------------------
% Available options: circle|rectangle,edge/noedge,left/right
% \photo{./examples/profile.png}
% \name{Ben Bubnick}{}
% \position{Analytics Training Specialist{\enskip\cdotp\enskip}Adjunct Professor}
% \address{3419 Superior Park Drive, Cleveland Heights, Ohio, 44118, U.S.}

\email{ben.bubnick@gmail.com}
%\homepage{www.benbubnick.com}
% \github{bubnicbf}
% \linkedin{benbubnick}
% \gitlab{gitlab-id}
% \stackoverflow{SO-id}{SO-name}
% \twitter{@twit}
% \skype{skype-id}
% \reddit{reddit-id}
% \extrainfo{extra informations}

% \quote{``10+ years experience physics/astronomy/analytics lecturer"}


%-------------------------------------------------------------------------------
%	LETTER INFORMATION
%	All of the below lines must be filled out
%-------------------------------------------------------------------------------
% The company being applied to
\recipient
  {Teaching Statement}
  {Ben Bubnick}
% The date on the letter, default is the date of compilation
\letterdate{\today}
% The title of the letter
% \lettertitle{Job Application for  Adjunct Faculty, Physics}
% How the letter is opened
\letteropening{Interactive Methods in Teaching of Physics}
% How the letter is closed
% \letterclosing{Sincerely,}
% Any enclosures with the letter
% \letterenclosure[Attached]{Resume}
% \letterenclosure[Attached]{Resume, Teaching Statement}


%-------------------------------------------------------------------------------
\begin{document}

% Print the header with above personal informations
% Give optional argument to change alignment(C: center, L: left, R: right)
% \makecvheader[R]

% Print the footer with 3 arguments(<left>, <center>, <right>)
% Leave any of these blank if they are not needed
\makecvfooter
  {\today}
  {Ben Bubnick~~~·~~~Teaching Statement}
  {}

% Print the title with above letter informations
\makelettertitle

%-------------------------------------------------------------------------------
%	LETTER CONTENT
%-------------------------------------------------------------------------------
\begin{cvletter}

Throughout my education and career, interactive teaching methods have become very popular, and its use brings about significantly improved results than the use of traditional methods.  I use a mix of methods such as Peer Instruction, Interactive Lecture Demonstration, and Just-in-time-teaching (JiTT), etc.  The essentials of the approach that I take in the classroom are to address: identifying preconceptions in the minds of students, avoiding and overcoming misconceptions, intentional use of mental processes such as as assimilation and accommodation, cognitive conflict as a trigger for changes in thought structures, simpler and more qualitative learning experiments, exercises to improve understanding, and explicitly making the connection between formalism and the real world.  

Of course, the necessity of interactive methods stems from the change in higher education as the meaning of the word “to know” has changed from “be able to remember” to “be able to find information and use it”--in other words, teaching today's students to employ critical thinking.  My experience in teaching physics has shown that an increased focus on teaching and learning experiments and the use of qualitative (problem) activities enables students to solve problems and to look for new ways to discover knowledge, and in the process produces an enthusiasm for the subject within the student.  And to that end, I use interactive methods that enhances students level of theory of physics while at the same time interconnecting the concepts of physics with everyday life.  Let's take a look at an example lecture using these methods:

\begin{enumerate}

	\item Preparatory phase: I set up basic physical concepts I want to explain in the lesson. I prepare a block of 5–10 minutes for each concept. The concept is being dealt with within this frame, and I also present few physical issues related to the given concept.

	\item Dealing with the concept: My lecture during the 5–10 minute block deals with the important physical concept (or concepts).
	
	\item Assignment of a problem: I follow with a presentation of a problem task connected to the concept. The problem task can be, assigned in several ways.  For example:
	
	\begin{itemize}
		\item Task assigned in the form of a text and solved theoretically (e.g. How does a ship’s draught change after shipping out from a river into the sea?).
		\item Task assigned in the form of a text and solved via traditional experiment / via computer aided experiment as is often the case in a Physics Lab.
		\item Task assigned via unfinished experiment (video-experiment, simulation, applet) – problem is what the experiment result will be (task solved theoretically, in the end the experiment is carried out / video-experiment or simulation to verify the theory is played).
		\item Task assigned via finished experiment (video-experiment, simulation, applet) – problem is physical reasoning of the course or result of experiment (task solved via video-analysis).
	\end{itemize}
	
	\item Problem solving: After I have introduced the problem a class discussion follows.  Larger classes can break up into groups during this phase. Under the my supervision as the teacher, students discuss possible solutions of the given problem. I gradually write the solutions on the board.  Students write incorrect solutions (elimination of misconception) and the correct solution in their worksheets.  Sometimes it is possible that the task is open and within the discussion more correct explanations of the given problem are possible.  Students write into their worksheets all solutions including the physical reasoning why the solution is correct or incorrect. In some cases, at the end the teacher carries out a verifying experiment, which shows whether the answer that seemed to be correct after the physical reasoning is really correct.
	
	\item Feedback: The final phase is the evaluation of the given concept. Feedback is carried out on several levels. I evaluate what extent the students were involved into solving individual tasks (from the viewpoint of intrinsic and extrinsic motivation but also from the aspect of the difficulty of individual tasks). An important part is whether the students improve in searching for correct answers for problem situations connected to the same concept.  Evaluations are made continuously, and changes during the semester are made. 
	
\end{enumerate}


\end{cvletter}

\end{document}
